\documentclass[paper=letter, fontsize=11pt]{scrartcl}

\usepackage[T1]{fontenc} % Use 8-bit encoding that has 256 glyphs
\usepackage{fourier} % Use the Adobe Utopia font for the document - comment this line to return to the LaTeX default
\usepackage[english]{babel} % English language/hyphenation
\usepackage{amsmath,amsfonts,amsthm} % Math packages
\usepackage{graphicx}
\graphicspath{{Waveform/}}
\usepackage{lipsum} % Used for inserting dummy 'Lorem ipsum' text into the template
\usepackage{hyperref}

\usepackage{sectsty} % Allows customizing section commands
\allsectionsfont{\centering \normalfont\scshape} % Make all sections centered, the default font and small caps

\usepackage{fancyhdr} % Custom headers and footers
\pagestyle{fancyplain} % Makes all pages in the document conform to the custom headers and footers
\fancyhead{} % No page header - if you want one, create it in the same way as the footers below
\fancyfoot[L]{} % Empty left footer
\fancyfoot[C]{} % Empty center footer
\fancyfoot[R]{\thepage} % Page numbering for right footer
\renewcommand{\headrulewidth}{0pt} % Remove header underlines
\renewcommand{\footrulewidth}{0pt} % Remove footer underlines
\setlength{\headheight}{13.6pt} % Customize the height of the header

\numberwithin{equation}{section} % Number equations within sections (i.e. 1.1, 1.2, 2.1, 2.2 instead of 1, 2, 3, 4)
\numberwithin{figure}{section} % Number figures within sections (i.e. 1.1, 1.2, 2.1, 2.2 instead of 1, 2, 3, 4)
\numberwithin{table}{section} % Number tables within sections (i.e. 1.1, 1.2, 2.1, 2.2 instead of 1, 2, 3, 4)

\setlength\parindent{0pt} % Removes all indentation from paragraphs - comment this line for an assignment with lots of text

%----------------------------------------------------------------------------------------
%	TITLE SECTION
%----------------------------------------------------------------------------------------

\newcommand{\horrule}[1]{\rule{\linewidth}{#1}} % Create horizontal rule command with 1 argument of height

\title{	
\normalfont \normalsize 
\textsc{University of California Irvine} \\  % Your university, school and/or department name(s)
\textsc{Course: Introduction to Digital Logic Lab (31L) Fall 2015} \\ [25pt]
\horrule{0.5pt} \\[0.4cm] % Thin top horizontal rule
\huge Final Project Report - Group 33\\ % The assignment title
\horrule{2pt} \\[0.5cm] % Thick bottom horizontal rule
}

\author{Kelvin Phan \\ 51197373
	\and
	Patrick Skoury \\ 75202200
	\and
	Aaron Liao \\ 90811748
	\and
	Shawn Howen \\ 60029119
	\and
	Andrew Mehta \\ 44592437
}

\date{\large\today} % Today's date

\begin{document}

\maketitle % Print the title

%----------------------------------------------------------------------------------------
%	PROBLEM 1
%----------------------------------------------------------------------------------------

\section{Direct Memory Access (DMA)}

\subsection{Architecture}
\begin{flushleft}
	The DMA's primary purpose is to transfer instruction address and data from the main memory to the processor's instruction memory independently of the processor. In our implementation, the DMA operates whenever the reset signal is high. Though there is only one main memory source, we still implement according to design, having a handshake between the memory and DMA. The memory sends a request signal to the DMA when data is ready to be transferred, and does so when it receives back an acknowledgment signal. The DMA in turn passes the instructions to the instruction memory where it can be retrieved and parsed the by the controller. \\[20pt]
\end{flushleft}


%------------------------------------------------
\section{Creating the Instruction Memory}

\subsection{Architecture}
\begin{flushleft}
	Our design for the instruction memory implemented the behavioral style of architecture used for sequential coding. We defined our input and output ports as follows: \\
	-Address Bus: 6 bit STD\_LOGIC\_VECTOR (IN)\\
	-Data Bus: 32 bit STD\_LOGIC\_VECTOR (INOUT)\\
	-Read/Write Bar: 1 bit IN\\
	-Output Enable Bar: 1 bit IN \\[10pt]
	
	Within the architecture, we used the behavioral model of coding, which included a PROCESS statement. Before our PROCESS, we created a subtype "word". "Word" is a 32 bit STD\_LOGIC\_VECTOR that is used in our array. We also created a type "mem" which is an array that holds 64 "word"s. Hence, 64x32. Our PROCESS statement had a sensitivity list that depended on "a" (address bus), "d" (data bus), "rwb" (read/write bar), and oeb (output enable bar). We then created the variable "mem1" which was type "mem" (our 64x32 array type). The IF statement checked if "oeb" and "rwb" were both 1. If so, it would assign the value in the mem1 array at the address "a" to our data bus "d". Or IF "rwb" was 0, then we would assign "d" the value at the address "a" in our mem1 array. This concludes the process. We fed in <NUMBER TESTED> random inputs to <PORTS TESTED> after <NUMBER OF ROUNDS> rounds. The test bench section of the report specifies how we tested our components in further detail. \\[20pt]      
\end{flushleft}

%------------------------------------------------
\section{Creating the Sign Extender}

\subsection{Architecture}
\begin{flushleft}
	Our design for the sign extender implemented the structural style of architecture. We defined our input and output ports as follows:\\
	-Reset: 1 bit IN
	-Input: 16 bit STD\_LOGIC\_VECTOR (IN)\\
	-Output: 32 bit STD\_LOGIC\_VECTOR (OUT)\\[10pt]
	
	Within the architecture we used the structural model of coding. We made it as simple as possible, all in one line. It was simply this: When the system isn't being reset (rst='0') update the output to be a 32 bit, sign extended version of the input. This concludes the architecture. \\[20pt]      
\end{flushleft}

\subsection{Errors Encountered}

\subsubsection{Simulation}
\begin{verbatim}

\end{verbatim} 

Our intention was to convert each parity bit into an integer, multiply each by its position, and add the integers up in order to determine the error location. The problem was that subtracting "err\_loc - 1" and adding "err\_loc + 1" resulted in a number that was out of bounds. In order to solve this error, we concatenated the parity bits into the vector "syndrome" and used that signal to find the error location. Rather than using a shortcut, we listed out all 12 cases for "data\_temp," which was described in the Architecture of our decoder. Then, we could set the output of the decoder to "data\_temp" without the parity bits.\\[20pt] 

%-------------------------------------------------------------
\section{Creating the Testbench}

\subsection{Architecture}
\begin{flushleft}
	Our design for the testbench, written in SystemVerilog, allows us to verify the functionality of our 
	         
\end{flushleft}



\end{document}
